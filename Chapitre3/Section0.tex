\section{Apprentissage artificiel}

% \comment{}{Présentation plus générale s'inspirant de http://fr.wikipedia.org/wiki/Apprentissage_supervis\%C3\%A9. Note qu'il existe aussi un apprentissage non-supervisé qu'on ignore ici.}

D'un point de vue informatique, un algorithme d'apprentissage peut être vu comme un objet doté de 2 types de paramètres:
\begin{itemize}
 \item des paramètres dépendant des données qu'on va estimer à partir de données d'apprentissage notés $\theta$
 \item des paramètres fixés à la création de l'objet appelés hyper-paramètres
\end{itemize}
Son utilisation se fait en 2 phases:
\begin{itemize}
 \item une phase d'apprentissage (voir~\autoref{fig:Apprentissage_Machine}) au cours de laquelle on va estimer les paramètres $\theta$ grâce à 2 jeux de données:
 
  	\begin{figure}[h]
  		\centering
  		\includegraphics[scale = 0.25]{images/Diagramme1}
  		\caption{Représentation de la phase d'apprentissage.}
  		\label{fig:Apprentissage_Machine}
  	\end{figure}
  	
 une matrice $X_{train}$ qui représente les données et une matrice $y_{train}$ qui représente les résultats correspondant aux données.
 On cherche en général des paramètres qui permettent d'optimiser un certain critère, notés $\theta^{*}$.
 Cependant, l'algorithme ne fournit pas en général les paramètres optimaux mais une approximation notée $\hat{\theta}$.
 
 \item une fois l'apprentissage effectué, on peut utiliser l'algorithme pour prédire les labels $y_{pred}$ associés à de nouvelles données $X_{test}$ (voir~\autoref{fig:Apprentissage_Machine_Prediction}).
 
 	\begin{figure}[h]
 		\centering
 		\includegraphics[scale = 0.25]{images/App_Mach_Prediction}
 		\caption{Utilisation d'un algorithme pour la prédiction de nouvelles données.}
 		\label{fig:Apprentissage_Machine_Prediction}
	 \end{figure}
\end{itemize}

Le critère utilisé pour choisir les paramètres lors de la phase d'apprentissage est le critère qui permet de faire le nombre d'erreur de classification
le plus petit possible lors de la phase de test (bien qu'on n'ait pas encore les données de test).

Il peut arriver qu'un algorithme donne de très bons résultats sur l'ensemble d'apprentissage mais de très mauvais résultats sur l'ensemble de test.
Ce péhnomène est appelé sur-apprentissage (ou apprentissage par cœur) doit être évité.

\todo[inline]{Évoquer qu'on doit linéariser les images}
