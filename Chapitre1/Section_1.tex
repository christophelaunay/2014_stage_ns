\section{Neuropsin}

NeuroSpin est une grande infrastructure de neuro-imagerie située à l’intérieur du CEA de Saclay. Ce centre possède des imageurs à résonance magnétique à champ intense et vise à repousser les limites actuelles de l’imagerie cérébrale. Tout d’abord, on ne peut parler de NeuroSpin sans aborder le bâtiment en lui-même. En effet, ce bâtiment très impressionnant, imaginé par Claude VASCONI, a mis deux ans à sortir de terre. Les travaux ont commencé en janvier 2005 et les locaux ont commencé à être utilisés depuis le 1er janvier 2007. Ce bâtiment, fait de béton, d’acier et de verre, forme un ensemble linéaire composé de deux édifices parallèles, de part et d’autre d’une nef centrale de 135 mètres de long, appelée encore \og{} galeria \fg{}. Cette nef, véritable colonne vertébrale de l’édifice, est conçue comme un espace de rencontres entre les différentes équipes de chercheurs. NeuroSpin s’étend sur environ 11 400m².

L’édifice Ouest, dont la façade est composée de structures sinusoïdales entièrement métalliques, est développé uniquement en rez-de-chaussée et contient les ensembles de recherche clinique, de recherche pré-clinique (des lits sont installés pour accueillir des patients) et les salles des aimants abrités dans des alcôves magnétiquement confinées.

Un sous-sol partiel est réservé à l’usage des locaux techniques et de stockage.
Une voie d’accès raccordée à l’entrée Est du Centre dessert le bâtiment et le parking extérieur de 100 places.

Le centre NeuroSpin est dirigé par Dr. Denis LE BIHAN, élu à l'académie des sciences et chevalier de l'Ordre national du mérite.

