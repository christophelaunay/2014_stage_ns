\section{Les objectifs}

NeuroSpin a été créé pour mieux comprendre le cerveau en cartographiant de plus en plus précisément les zones qui sous-entendent les fonctions cognitives.
Ces images permettent de distinguer les assemblées de neurones et les processus mentaux mis en jeu dans le langage, la mémoire, le calcul, la préparation à l’action, l’apprentissage de la lecture voire même la conscience. 

Le second objectif consiste à comprendre le cheminement et le mode de traitement de l’information dans le cerveau.
Le troisième vise à élucider le « code neural », un code qui doit reposer sur une organisation très structurée dans l’espace des assemblées de neurones.
Le dernier objectif de NeuroSpin est de mieux comprendre les pathologies cérébrales pour mieux les reconnaître, les prévenir et les traiter.

Pour ce faire le centre possède des imageurs à résonance magnétique élevés de 3 et 7 Tesla et prévoit la construction d’un imageur à 11,7T. Par comparaison, les IRM présents dans les hôpitaux exercent un champ magnétique de 1,5 à 3T. Pourquoi des aimants toujours plus puissants ? Pour pouvoir améliorer la sensibilité des imageurs et ainsi obtenir des résolutions spatiales et temporelles optimales !

La construction de l’aimant à 11,7T sera une première mondiale. Cette construction se fait dans le cadre d’un partenariat franco-allemnand.

Pour résumé, NeuroSpin est une infrastructure moderne, active, dans l’innovation avec son futur imageur de 11,7T qui occupe une place de rang international dans la recherche concernant le cerveau. De nombreux projets sont en cours de développement au niveau national mais aussi européen avec le partenariat franco-allemand.
