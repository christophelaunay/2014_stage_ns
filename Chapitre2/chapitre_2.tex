\chapter{Présentation du projet et cahier des charges}

% Étude du trouble bipolaire par l'imagerie de diffusion.
Le projet consiste à déterminer, à partir d'un examen d'Imagerie par Résonance Magnétique (IRM), si des patients sont atteints de trouble bipolaire ou non.
La motivation pour cette étude est que ces examens sont indolores et rapides.
Pour cela, nous utilisons des techniques d'apprentissage artificiel (appelé \textit{Machine Learning} en anglais).
En plus de permettre la classification des patients à partir d'un examen IRM, on peut, grâce à ces techniques, étudier quelles régions du cerveau sont impliquées dans la maladie et ainsi
obtenir des marqueurs biologiques de la maladie (\textit{biomarkers} en anglais).

Nous avons travaillé avec une modalité d'IRM particulière appelée IRM de diffusion (\textit{Diffusion Weighted Imaging (DWI)} en anglais)
que nous présentons brièvement dans la~\autoref{sec:dwi}.

\section{Les troubles bipolaires : explication}

Pour bien comprendre la nature du projet, il faut savoir ce que sont les troubles bipolaires.
Une bipolarité est un trouble mental qui influe sur l'humeur caractérisé par l'oscillation entre des périodes de dépression et des périodes d'élévation de l'humeur (manie ou hypomanie) avec entre les deux,
des périodes d'humeur \og{} normale \fg{} (euthymie).

\comment{Mathieu}{Tu peux mettre quelques chiffres (nombre de personnes  touchées) et pourquoi c'est grave. Voir http://fr.wikipedia.org/wiki/Trouble\_bipolaire et http://en.wikipedia.org/wiki/Bipolar\_disorder}
\section{DWI : Diffusion Weighted Imaging} \label{sec:dwi}

L'IRM de diffusion est une technique basée sur l'imagerie par résonance magnétique (IRM).
Elle permet de mesurer en chaque point du volume imagé la distribution des directions de diffusion des molécules d'eau à l'intérieur du cerveau.
La diffusion des molécules d'eau est contrainte par les tissus environnants, cette modalité d'imagerie permet d'obtenir indirectement la position, l'orientation et l'anisotropie des structures fibreuses, notamment les faisceaux de matière blanche du cerveau.
L'hypothèse qui sous-tend le projet est que des changements dans la matière blanche peuvent indiquer des changements pathologiques.

\section{Présentation du projet}

\comment{Mathieu}{Cette partie n'est pas très claire. Je pense qu'il ne faut pas parler ici des échantillons d'apprentissage et de tests.
                  En revanche, tu peux dire ce qui était présent au début du projet: données cliniques et images provenant de différents centres, avec les données
                  des identifiants différents et un ordre différent.
                  Ainsi, tu justifie la première partie du cahier des charges.
                  De plus, tu peux dire qu'on utilise un modèle doté de nombreux paramètres et donc qu'on va devoir essayer beaucoup de combinaisons.
                  Comme chaque évaluation est lourde (validation croisée), ça nécessite l'emploi d'un cluster et des logiciels développés à NS pour ça.
                  Il faut aussi évoquer les différentes hypothèses (différents masques et images squeletissées).
                  La partie 4 donne des détails qui peuvent être intéressants ici.
                  Voir les remarques en section 4}
Le projet consiste à étudier une population de 200 sujets dont une partie est atteinte de trouble bipolaire et l'autre est constitué de témoins.
% Le but étant d'apprendre à la machine à reconnaître les sujets bipolaires des sains.
Le principe est de créer deux groupes: un qui servira pour l'apprentissage et l'autre de test.
À partir de l'échantillon d'apprentissage, l'algorithme estime une carte des poids des voxels (les pixels de l'image IRM)
Grâce à cet apprentissage, on peut classer les sujets de l'échantillon de test et comparer ce résultat aux statuts cliniques.

Ainsi, nous avons le plan d'ensemble du projet, procédons maintenant par étapes en commençant par présenter le cahier des charges.

\section{Le cahier des charges}

Une présentation des différentes étapes qui vont mener à la prédiction des résultats :

\begin{itemize}
 \item Construction de la population: fusionner la liste des images et les données cliniques
 \item Mise en forme des images et des variables d'intérêt pour leur utilisation par l'algorithme d'apprentissage
 \item Paramétrer et lancer les calculs sur le cluster
 \item Interprétation
\end{itemize}

Pour continuer sur la mise en œuvre de ce projet, un peu de théorie concernant les méthodes utilisés. 
