\chapter{Présentation du projet et cahier des charges}

% Étude du trouble bipolaire par l'imagerie de diffusion.
\comment{Mathieu}{J'aurais mis cette partie en présentation du projet (donc mélanger avec la section 3).}
Le projet consiste à déterminer, à partir d'un examen d'imagerie, si des patients sont atteints de trouble bipolaire ou non.
La motivation pour cette étude est que ces examens sont indolores et rapides.
De plus, on peut ainsi étudier quelles régions du cerveau sont impliquées dans la maladie.
Pour cela, nous utilisons des techniques d'apprentissage artificiel (appelé \textit{Machine Learning} en anglais).
Les images sur lesquelles nous avons travaillé ont été acquises selon la technique de 
\textit{Diffusion Weighted Imaging (DWI)}.
DWI est une technique d'imagerie IRM qui sert principalement à imager les flux des molécules d'eau à l'intérieur du cerveau.
Par cette technique, nous obtenons des mesures du cerveau dépendant de la matière blanche. 


\section{Les troubles bipolaires : explication}

Pour bien comprendre la nature de la recherche, il faut savoir ce que sont les troubles bipolaires. Une bipolarité est un trouble mental qui influe sur l'humeur, oscillant entre des périodes de dépression et d'hypomanie (élévation de l'humeur) avec entre les deux, des périodes d'humeur "normale" (euthymie).


\section{DWI : Diffusion Weighted Imaging} \label{sec:dwi}

L'IRM de diffusion est une technique basée sur l'imagerie par résonance magnétique (IRM).
Elle permet de mesurer en chaque point du volume imagé la distribution des directions de diffusion des molécules d'eau à l'intérieur du cerveau.
La diffusion des molécules d'eau est contrainte par les tissus environnants, cette modalité d'imagerie permet d'obtenir indirectement la position, l'orientation et l'anisotropie des structures fibreuses, notamment les faisceaux de matière blanche du cerveau.
L'hypothèse qui sous-tend le projet est que des changements dans la matière blanche peuvent indiquer des changements pathologiques.

\section{Présentation du projet}

% \comment{Mathieu}{Cette partie n'est pas très claire. Je pense qu'il ne faut pas parler ici des échantillons d'apprentissage et de tests.
%                   En revanche, tu peux dire ce qui était présent au début du projet: données cliniques et images provenant de différents centres, avec les données
%                   des identifiants différents et un ordre différent.
%                   Ainsi, tu justifie la première partie du cahier des charges.
%                   De plus, tu peux dire qu'on utilise un modèle doté de nombreux paramètres et donc qu'on va devoir essayer beaucoup de combinaisons.
%                   Comme chaque évaluation est lourde (validation croisée), ça nécessite l'emploi d'un cluster et des logiciels développés à NS pour ça.
%                   Il faut aussi évoquer les différentes hypothèses (différents masques et images squeletissées).
%                   La partie 4 donne des détails qui peuvent être intéressants ici.
%                   Voir les remarques en section 4}

Afin d'étudier la maladie, une base de données contenant des patients atteints de bipolarité et des sujets sains (sujets contrôle) a été constituée
en collaboration entre plusieurs hôpitaux en France, en Allemagne et aux États-Unis.
Ces sites nous ont fourni les données cliniques de chaque sujet ainsi que leur IRM respectif.
% Compte tenu du fait que les données proviennent de 3 sites différents, les identifiants des sujets ne correspondent pas toujours aux a ceux des images IRM ou bien l'ordre est différent ou bien il y a des données manquantes, etc... 
% Suite à la résolution de ces différents problèmes, nous étudions ces données avec un modèle estimateur qui possède 5 hypers-paramètres (K, $\alpha$, l1, l2, tv). Chacun de ces paramètres possède plusieurs valeurs et chacune des combinaisons de ces 5 valeurs seront testées ce qui correspond a 2475 combinaisons. 

Nous voulons évaluer un algorithme d'apprentissage développé à Neurospin.
Cet algorithme possède de nombreux paramètres.
Nous voulons donc trouver la meilleure combinaisons de ces paramètres.

Comme nous avons beaucoup de combinaisons à tester et que la méthode d'évaluation est lourde,
nous utiliserons un cluster et des logiciels développés par Neurospin afin de paralléliser l'évaluation.

Nous voulons de plus tester plusieurs hypothèses scientifiques telles que:
\begin{itemize}
	\item l'influence d'autres variables telles que le sexe, l'âge ou le scanner IRM utilisé
	\item la restriction a une partie du cerveau afin de focaliser notre analyse sur des régions précises du cerveau
	\item l'utilisation de pré-traitement supplémentaire (squeletisation) sur les images
\end{itemize}

% Le sujet de ce programme repose sur différentes hypothèses : 
% \begin{enumerate}
% 	\item les images initiales et une matrice avec le sexe et l'age auquel l'image a été prise l'IRM sans intercept.
% 	\item les images initiales avec une matrice au mêmes covariables avec l'intercept. 
% 	\item les deux premières hypothèses mais avec des images squeletonisées.
% 	\item Les images de bases avec une covariable en plus : les sites ont été rajoutés
% 	\item un masque tronqué d'une partie du cerveau afin de focaliser notre analyse sur des régions précises du cerveau.  
% \end{enumerate}
% 
% les hypothèses sans intercept et sans sites ne seront pas présentés dans ce rapport pour raison de résultat insatisfaisant. 

\section{Le cahier des charges}

Une présentation des différentes étapes qui vont mener à la prédiction des résultats :

\begin{itemize}
 \item Construction de la population: fusionner la liste des images et les données cliniques
 \item Mise en forme des images et des variables d'intérêt pour leur utilisation par l'algorithme d'apprentissage
 \item Paramétrer et lancer les calculs sur le cluster
 \item Interprétation
\end{itemize}

Pour continuer sur la mise en œuvre de ce projet, un peu de théorie concernant les méthodes utilisés. 
