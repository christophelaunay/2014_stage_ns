\section{Les troubles bipolaires : explication}

Pour bien comprendre la nature du projet, il faut savoir ce que sont les troubles bipolaires.
Une bipolarité est un trouble mental qui influe sur l'humeur. Il est caractérisé par l'oscillation entre des périodes de dépression et des périodes d'élévation de l'humeur (manie ou hypomanie) avec entre les deux,
des périodes d'humeur \og{} normale \fg{} (euthymie).

Cette maladie touche 1 à 2\% de la population mondiale. C'est  la maladie neurologique la plus morbide en terme de suicide. Le risque suicidaire est trente fois supérieur à celui de la population générale et 15 à 19\% des patients atteints de cette maladie \og{} réussissent \fg{} leur suicide. À cette mortalité par suicide, vient s’ajouter la mortalité liée à de nombreux autres facteurs :  alcoolisme, mauvaise hygiène de vie, diabète, etc... Du fait des addictions diverses et des troubles du comportement, il semble qu’un individu bipolaire non traité ait en moyenne une espérance de vie inférieure de vingt ans à l'espérance de vie de la population générale
L'élément le plus grave de cette maladie est la latence du diagnostic qui peut prendre plusieurs années et retarder le commencement d'un traitement adéquat. 
