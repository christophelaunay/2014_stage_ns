\section{Les sujets}

Cette étape va concerner l'initialisation de nos sujets. Comment ils ont été choisi, quels sont les spécificités, etc...

\subsection{Le choix des participants}

Les participants a ce projet ont été recruté dans trois sites différents : 
\begin{itemize}
	\item Assistance Publique-Hôpitaux de Paris Hôpital Henri Mondor-Albert Chenevier à Créteil, Fernand Widal-Lariboisière à Paris, France
	\item Western Psychiatric Institute and Clinic in Pittsburgh, Pennsylvania
	\item Central Institute for Mental Health in Mannheim, Germany, une fondation publique associée a l'université de Heidelberg.
\end{itemize}
les sujets témoins ont été recrutés parmi des contrôles, des annonces médiatiques ou encore dans des bureaux d'enregistrement parmi les trois sites. Ils ont tous été soumis à des tests reconnus pour les troubles mentaux et non aucun membre de leur famille sujet a ces mêmes troubles. Un autre critère de sélection est qu'aucun des participants n'a subi de traumatisme neurologique, n'ai une contre indication pour l'IRM.


\subsection{L'acquisition des données}

Toute les données ont été acquises par le même logiciel d'acquisition avec une machine IRM 3T qui ont toutes été paramétrés de la même manière sur les 3 sites. Par la suite, les images ont été normalisés et corrigés si besoin à l'aide de logiciel libre.

\todo{inclure un petit schéma concernant le prétraitement des données}

\subsection{Création du fichier de la population} 

Au commencement, nous avons 3 fichiers :
\begin{itemize}
	\item Le fichier qui contient les images de diffusion : \filename{all\_FA.nii.gz}
	\item Un autre contenant les identifiants de chaque sujet qui correspondent aux images : \filename{ID.tbss}
	\item Un dernier fichier qui correspond aux données cliniques de chaque sujet: \filename{BD\_clinic.xlsx} 
\end{itemize}

Avant tout, il faut s'assurer que les identifiants contenus dans le \filename{ID.tbss} soient les mêmes que dans notre fichier clinique.
Une fois cela fait, nous sélectionnons les données cliniques qui nous intéressent pour l'étude en question. C'est à dire les données suivantes : 
\begin{itemize}
	\item L'identifiant du sujet
	\item l'état du patient: si il est bipolaire ou non
	\item l'âge auquel l'IRM a été effectué
	\item le sexe du sujet
	\item le site dans lequel l'acquisition a été faite 
\end{itemize}

À la fin de toutes ces étapes, ces données sont enregistrés dans un nouveau fichier que nous appelons \filename{population.csv} que nous utiliserons tout du long du projet.


Une fois toutes ces données récupérées, les images seront traitées. 