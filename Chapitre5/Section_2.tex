\section{Résultats de l'apprentissage selon l'hypothèse 2 }

Pour rappel : L'hypothèse 2 est une analyse des IRM avec une troncature des parties du cerveau qui ne nous intéressent pas. Cette troncature a été faite après lecture et étude des publications déjà parues dans la littérature scientifique avec l'aide du Dr Houenou qui nous a aidé pour décider les zones du cerveau à éliminer. 
En fait, les parties éliminées sont toutes situées à l'arrière du cerveau ( cervelet, bulbe, tronc cérébral principalement). 
Les covariables sont les mêmes que pour l'hypothèse 1.
\\
Les 10 jeux de paramètres pour lesquels nous avons obtenus les meilleurs résultats sont rassemblés dans le tableau 2.

\begin{tabular}{|l|l|l|c|r|}
	\hline
	jeu & paramètres & Sensibilité & Spécificité & Moyenne \\
	\hline
	1 & (0.05, 0.196, 0.196, 0.601, -1.0) & 0.593 & 0.703 & 0.648 \\
	2 & (0.5, 0.009, 0.0898, 0.9, 100000.0) & 0.755 & 0.527 & 0.641 \\
	3 & (0.05, 0.098, 0.0, 0.9, -1.0) & 0.604 & 0.666 & 0.635 \\
	4 & (1.0, 0.009, 0.089, 0.9, -1.0) & 0.639 & 0.629 & 0.634 \\
	5 & (0.1, 0.0095, 0.941, 0.05, 100000.0) & 0.639 & 0.620 & 0.629 \\
	6 & (1.0, 0.007, 0.693, 0.304, -1.0) & 0.639 & 0.620 & 0.629 \\
	7 & (0.05, 0.45, 0.45, 0.1, -1.0) & 0.546 & 0.712 & 0.629 \\
	8 & (1.0, 0.00095, 0.949, 0.05, 100000.0) & 0.709 & 0.546 & 0.627 \\
	9 & (0.05, 0.267, 0.0299, 0.701, -1.0) & 0.569 & 0.685 & 0.627 \\
	10 & (0.05, 0.009, 0.0899, 0.9, 100000.0) & 0.744 & 0.509 & 0.626 \\
	\hline
	
\end{tabular}
\\
Ces résultats présentent plus de variabilité que ceux obtenus avec l'hypothèse précédente.
La sensibilité allant de 54\% à 75\% et la spécificité de 50\% et 71\%. Là encore, les jeux de paramètres avec la meilleure sensibilité (n°2) est le second moins spécifiques alors que les paramètres les plus spécifiques (n°7)
sont les moins sensibles.  
Cette variation n'est pas souhaitable car nous ne voulons pas avoir de doute lorsque l'on diagnostique ce genre de maladie. 
\\
La classification moyenne obtenue varie de 62\% à 64\% ce qui reste non significatif et non différent de l'hypothèse 1 d'où l'hypothèse 3.  


