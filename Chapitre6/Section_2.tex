\section{Le bilan technique}

\subsection{Python}

La première chose que j'ai dû apprendre durant mon stage est un nouveau langage: python. 
Ce nouveau langage de haut niveau tire ses bases de Matlab, un outil puissant pour les mathématiques appliqués. 
Il possède un interpréteur comme Java, de puissants outils pour le calcul (en particulier pour les calculs matriciels) ce qui était parfait pour notre recherche.
\todo[inline]{Proposition de reformulation: C'est un langage interprété de haut niveau.
Il possède de puissants outils pour le calcul en particulier pour les calculs matriciels (inspiré de Matlab, logiciel pour les mathématiques appliqués) ce qui est nécessaire pour notre recherche.\\
Tu peux rajouter un blabla du style "c'est un lanage très utilisé dans l'industrie donc je suis super heureux de l'avoir appris"}


\subsection{La recherche}

La deuxième chose que j'ai apprise est la recherche. On se fait toujours une idée de la recherche sans vraiment savoir ce que c'est, comment on procède, quelles sont les étapes, etc... 
À ce jour, grâce à ce projet, j'ai pu participer à un projet de recherche. La première étape est de partir sur une hypothèse, analyser les résultats, émettre d'autres hypothèses réfléchies, les tester et recommencer.
\todo[inline]{On peut rajouter comparer avec la littérature}


Les résultats rapportés dans la littérature, basés sur l'imagerie anatomique rapportent des résultats équivalents aux nôtres (aux alentours de 65\%).
La nouveauté de notre étude est que nous utilisons l'imagerie de diffusion.
\todo[inline]{Je mettrais la phrase "ce qui prouve qu'il y a bien des marqueurs biologiques dans la diffusion et pas seulement au niveau anatomie" à la de la fin de la partie résultats. En fait, cette partie pourrait aller à la fin de la partie résultats.}
Cette découverte veut dire que dans le futur, 2 approches sont possibles pour étudier le trouble bipolaire :
\begin{itemize}
	\item Imagerie anatomie
	\item Imagerie de diffusion
\end{itemize}

Il est prévu qu'une publication soit rédigée à partir de ce travail car, acctuellement, les résultats que nous avons obtenus ont été trouvé sur des images anatomiques. 